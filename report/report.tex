\documentclass[12pt,a4paper]{scrartcl}
\usepackage[english]{babel}
\usepackage{url}
\usepackage{graphicx}

\title{\large{Highly Available, Distributed and Fault Tolerant Storage System} \\ \normalsize{Distributed Systems 2011}}
\author{Karsten Westra\\1693905 \and Edwin-Jan Harmsma\\1735535}

\begin{document}
\maketitle

\tableofcontents
\clearpage


% @karsten: ik heb 'problem statement' en 'state of the art' omgewisseld
% omdat ik state of the art meer bij solution vind horen...

\section{Context}
% basics principles of storage systems
% key->value, distributed file systems, memcache
% introduce fault-tolerance (replication) and scalability

\section{Problem statement}
% introduce all requirements of our system
% communication between different platforms

\section{State of the Art}
% introduce XOR idea (raid4 and raid5)
% asynchronous client-server, FIFO channels
% binary header-based protocol (platform independent)

\section{Solution details}
% overal architecture: layer architecture (see diagram other document)
\subsection{Storage service}
\subsection{Dictionary service}
\subsection{Freelist service}
\subsection{Front-end / proxy}
\subsection{Client}

\section{Results}
% discuss also the properties that are explained in the DS lectures

\section{Future improvements}
% system heavily relies on system clock (signing security) --> internal clock synchronization between components, is not implemented due the lack of time

% storage service:
% Remove blocking code between storageservice and parity server --> show blocking problem by sequence diagram, same for recovery
% Private key meganism, now every server has its own private key which is a big security issue --> only operation,offset,length/data are signed, NOT HOST/PORT!


\bibliographystyle{plain}
\bibliography{ref}
\nocite{*}

\end{document}
